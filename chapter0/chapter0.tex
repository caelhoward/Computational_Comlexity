\documentclass[11pt]{exam}
\usepackage{amsmath}
\usepackage{amsthm}
\usepackage{amssymb}
\newcommand{\myname}{Cael Howard} %Write your name in here
\newcommand{\myUCO}{} 
\newcommand{\myhwtype}{}
\newcommand{\myhwnum}{5} %Homework set number
\newcommand{\myclass}{Computational Complexity}
\newcommand{\mylecture}{}
\newcommand{\mysection}{}
\usepackage{tcolorbox}


% Prefix for numedquestion's
\newcommand{\questiontype}{Question}


% Use this if your "written" questions are all under one section
% For example, if the homework handout has Section 5: Written Questions
% and all questions are 5.1, 5.2, 5.3, etc. set this to 5
% Use for 0 no prefix. Redefine as needed per-question.
\newcommand{\writtensection}{0}

\usepackage{amsmath, amsfonts, amsthm, amssymb}  % Some math symbols
\usepackage{enumerate}
\usepackage{enumitem}
\usepackage{graphicx}
\usepackage{hyperref}
\usepackage[all]{xy}
\usepackage{wrapfig}
\usepackage{fancyvrb}
\usepackage[T1]{fontenc}
\usepackage{listings}
\usepackage{accents}

\usepackage{centernot}
\usepackage{mathtools}
\DeclarePairedDelimiter{\ceil}{\lceil}{\rceil}
\DeclarePairedDelimiter{\floor}{\lfloor}{\rfloor}
\DeclarePairedDelimiter{\card}{\vert}{\vert}


\setlength{\parindent}{0pt}
\setlength{\parskip}{5pt plus 1pt}
\pagestyle{empty}

\def\indented#1{\list{}{}\item[]}
\let\indented=\endlist

\newcounter{questionCounter}
\newcounter{partCounter}[questionCounter]

\newenvironment{namedquestion}[1][\arabic{questionCounter}]{%
    \addtocounter{questionCounter}{1}%
    \setcounter{partCounter}{0}%
    \vspace{.2in}%
        \noindent{\bf #1}%
    \vspace{0.3em} \hrule \vspace{.1in}%
}{}

\newenvironment{numedquestion}[0]{%
	\stepcounter{questionCounter}%
    \vspace{.2in}%
        \ifx\writtensection\undefined
        \noindent{\bf \questiontype \; \arabic{questionCounter}. }%
        \else
          \if\writtensection0
          \noindent{\bf \questiontype \; \arabic{questionCounter}. }%
          \else
          \noindent{\bf \questiontype \; \writtensection.\arabic{questionCounter} }%
        \fi
    \vspace{0.3em} \hrule \vspace{.1in}%
}{}

\newenvironment{alphaparts}[0]{%
  \begin{enumerate}[label=\textbf{(\alph*)}]
}{\end{enumerate}}

\newenvironment{arabicparts}[0]{%
  \begin{enumerate}[label=\textbf{\arabic{questionCounter}.\arabic*})]
}{\end{enumerate}}

\newenvironment{questionpart}[0]{%
  \item
}{}

\newcommand{\answerbox}[1]{
\begin{framed}
\vspace{#1}
\end{framed}}

\newtcolorbox{greybox}[1][]{
  colback=gray!10!white, % Light grey background
  colframe=black!50,    % Dark grey frame
  fonttitle=\bfseries,
  title=#1,
  sharp corners,
  rounded corners,
  leftrule=1pt,
  rightrule=1pt
}

\pagestyle{head}

\headrule
\header{\textbf{\myclass\ \mylecture\mysection}}%
{\textbf{\myname\ (\myUCO)}}%
{\textbf{\myhwtype\ \myhwnum}}

\begin{document}
\thispagestyle{plain}
\begin{center}
  {\Large \myclass{} \myhwtype{} \myhwnum} \\
  \myname{} \myUCO{} \\
  \today
\end{center}


\begin{numedquestion}
    For each of the following pairs of functions $f, g$ determine whether $f = o(g),\, g = o(f)$, or $f = \Theta(g)$ then find the first number $n$ such that $f(n) < g(n)$:
    \vspace{1em}

    a) $f(n) = n^2,\, g(n)=2n^2 + 100\sqrt{n}$\\
    b) $f(n) = n^{100}, g(n) = 2^{n/100}$\\
    c) $f(n) = n^{100}, g(n) = 2^{n^{n/100}}$\\
    d) $f(n)  \sqrt{n}, g(n) = 2^{\sqrt{\text{log}n}}$\\
    e) $f(n) = n^{100}, g(n) = 2^{(\text{log}n)^2}$\\
    f) $f(n) = 1000n, g(n) = n\text{log}n$
    \begin{greybox}
        a) $f = \Theta(g)$
        \vspace{1em}

        b) $f = o(g)$
        \begin{align*}
            n^{100} &< 2^{n/100}\\
            100\text{log}n &< n/100\\
            10000 &< n/\text{log}n\\
        \end{align*}
        c) $f = o(g)$\\
        d) $f = o(g)$\\
        e) $f = o(g)$\\
        f) $f = o(g)$
    \end{greybox}

    \begin{numedquestion}
      For each of the following recursively defined functions $f$, find a closed expression for a function $g$ such that $f(n) = \Theta(g(n))$, and prove that this is the case.
      \vspace{1em}

      \textbf{(a)} $f(n) = f(n-1) + 10$\\
      \textbf{(b)} $f(n) = f(n-1) + n$\\
      \textbf{(c)} $f(n) = 2f(n-1)$\\
      \textbf{(d)} $f(n) = f(n/2) + 10$\\
      \textbf{(e)} $f(n) = f(n/2) + n$\\
      \textbf{(f)} $f(n) = 2f(n/2) + n$\\
      \textbf{(g)} $f(n) = 3f(n/2)$\\
      \textbf{(h)} $f(n) = 2f(n/2) + O(n^2)$
      \begin{greybox}
        \textbf{(a)} For each $f(k)$, the total number of steps that needs to be done is $f(k-1) + O(1)$. Since it is a constant amount of work for $n$ values, $f(n) = \Theta(n)$ in this case.

        \textbf{(b)} For each $f(k)$, the total number of steps that needs to be done is $O(n)$. Since it is linear amount of work for $n$ values, $f(n) = \Theta(n^2)$

        \textbf{(c)} Expanded out, this function evaluates to $O(2^n)$, so $f = \Theta(2^n)$.

        \textbf{(d)} Since there is $O(1)$ work being done $\log n$ times, $f = \log n$

        \textbf{(e)} Since there is $O(k)$ work begin done $\log n$ times, where $k$ is the input of $f$, $f = \Theta{n}$ since $k$ is halving at every step

        \textbf{(f)} Since there is $O(n)$ work being done $\log n$ times, $f = \Theta(n\log n)$

        \textbf{(g)} Since $\log_2 3 > 0$, $f = \Theta(n^{log_2 3})$
        
        \textbf{(h)} Since $n^2 > 1$, $f = n^2$

      \end{greybox}
    \end{numedquestion}

    \begin{greybox}
      \textbf{(0.3)} The reason the machine does not break apart is due to the number of rotations needed from the first gear to move the final gear even a little. The number of turns to rotate the $k$th gear a single time is $50^k$. Therefore the total number of rotations the last gear does in one minute is $50^{13} / 212$ which is too large a number to notice any change in the last gears rotation. 
    \end{greybox}
\end{numedquestion}

\begin{numedquestion}
  Demonstrate that $e$ is an irrational number.

  \begin{proof}
  For sake of contradiction, assume that $e$ is a rational number, that is, that it can be written as $e=\frac{a}{b}$. Recall that $e$ can be written as follows:
  $$e = \sum_{n=0}^{\infty}\frac{1}{n!}$$
  Therefore, since $e = \frac{a}{b}$, we can write it as:
  $$\frac{a}{b} = \sum_{n=0}^{\infty}\frac{1}{n!}$$
  Since the summation iterates through all integers, we can split it into two separate summations, writing the ration $a/b$ as:
  $$\frac{a}{b} = \sum_{n=0}^{b}\frac{1}{n!} + \sum_{n=b+1}^{\infty}\frac{1}{n!}$$
  Since all of the terms in the first summation are multiples of all future terms in the summation, every single term can be written as a fraction with $b!$ in the denominator, meaning that the first summation can be written as:
  $$\frac{a}{b} = \frac{k}{b!} + \sum_{n=b+1}^{\infty}\frac{1}{n!}$$
  For some integer $k$. 
  \end{proof}
\end{numedquestion}





\end{document}

